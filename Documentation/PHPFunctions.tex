%%%%%%%%%%%%%%%%%%%%%%%%%%%%%%%%%%%%%%%%%
% University Assignment Title Page 
% LaTeX Template
% Version 1.0 (27/12/12)
%
% This template has been downloaded from:
% http://www.LaTeXTemplates.com
%
% Original author:
% WikiBooks (http://en.wikibooks.org/wiki/LaTeX/Title_Creation)
%
% License:
% CC BY-NC-SA 3.0 (http://creativecommons.org/licenses/by-nc-sa/3.0/)
% 
% Instructions for using this template:
% This title page is capable of being compiled as is. This is not useful for 
% including it in another document. To do this, you have two options: 
%
% 1) Copy/paste everything between \begin{document} and \end{document} 
% starting at \begin{titlepage} and paste this into another LaTeX file where you 
% want your title page.
% OR
% 2) Remove everything outside the \begin{titlepage} and \end{titlepage} and 
% move this file to the same directory as the LaTeX file you wish to add it to. 
% Then add \input{./title_page_1.tex} to your LaTeX file where you want your
% title page.
%
%%%%%%%%%%%%%%%%%%%%%%%%%%%%%%%%%%%%%%%%%

%----------------------------------------------------------------------------------------
% PACKAGES AND OTHER DOCUMENT CONFIGURATIONS
%----------------------------------------------------------------------------------------

\documentclass[12pt]{article}
\usepackage{parskip}
\usepackage{enumitem}
\usepackage{setspace}
\usepackage{graphicx}
\usepackage[margin=1.0in] {geometry}
\usepackage[none] {hyphenat}
\setlength{\parindent}{10ex}
\newcommand*{\SignatureAndDate}[1]{%
    \par\noindent\makebox[2.5in]{\hrulefill} \hfill\makebox[2.0in]{\hrulefill}%
    \par\noindent\makebox[2.5in][l]{#1}      \hfill\makebox[2.0in][l]{Date}%
}%

\begin{document}
\begin{titlepage}

\newcommand{\HRule}{\rule{\linewidth}{0.5mm}} % Defines a new command for the horizontal lines, change thickness here

\center % Center everything on the page
 
%----------------------------------------------------------------------------------------
% HEADING SECTIONS
%----------------------------------------------------------------------------------------

\textsc{\LARGE University Of Tennessee\\ Knoxville}\\[1.5cm] % Name of your university/college
\textsc{\Large Art Of Hiking Application}\\[0.5cm] % Major heading such as course name
%\textsc{\large Team \#13}\\[0.5cm] % Minor heading such as course title

%----------------------------------------------------------------------------------------
% TITLE SECTION
%----------------------------------------------------------------------------------------

\HRule \\[0.4cm]
{ \huge \bfseries Description of Tables}\\[0.4cm] % Title of your document
\HRule \\[1.5cm]
 
%----------------------------------------------------------------------------------------
% AUTHOR SECTION
%----------------------------------------------------------------------------------------

\begin{minipage}{0.4\textwidth}
\begin{flushleft} \large
\emph{Authors:}\\
Gabriel \textsc{Hanas}\\
Chris \textsc{Tester}\\
Robert \textsc{Moncrief}\\
Trevor \textsc{Jones}\\
Anthony \textsc{Stewart}\\
\end{flushleft}
\end{minipage}
~
\begin{minipage}{0.4\textwidth}
\begin{flushright} \large
\emph{Customer:} \\
Bradley \textsc{Vander Zanden} % Supervisor's Name
\end{flushright}
\end{minipage}\\[4cm]

% If you don't want a supervisor, uncomment the two lines below and remove the section above
%\Large \emph{Author:}\\
%John \textsc{Smith}\\[3cm] % Your name

%----------------------------------------------------------------------------------------
% DATE SECTION
%----------------------------------------------------------------------------------------

{\large \today}\\[3cm] % Date, change the \today to a set date if you want to be precise

%----------------------------------------------------------------------------------------
% LOGO SECTION
%----------------------------------------------------------------------------------------

%\includegraphics{Logo}\\[1cm] % Include a department/university logo - this will require the graphicx package
 
%----------------------------------------------------------------------------------------

\vfill % Fill the rest of the page with whitespace

\end{titlepage}
\tableofcontents


\newpage
\section{db\_connect}
\textbf{Usage:} no arguments \\
\textbf{Description:} a function that connects to the pre-defined database using the PHP PDO object. If the connection is successful, the connection is returned; otherwise, a boolean value of false is returned. In these cases the rest of the function/program calling db\_connect should stop execution.\\

\newpage
\section{isInDatabase}
\textbf{Usage:} sql, expected, id \\
sql - A query string that should be checking if some row already exists 
in the database. The query will be executed and the result and row count
compared. If the query fails for some reason, an error message will return
that it failed and the query string it attempted to execute. \\

expected - The number of rows expected to be returned by the query. Typically
either 0 or 1. \\

id - A value to be returned by the query. If this value is not passed
in as null, then in the SELECT statement the id must also be present. \\

\textbf{Description:} a function that checks if a certain row is in a table.
If id is passed in as null, then the return value will either be an error string, or it will be "Successful!". If id is not null, then the value in 
the SELECT statement will be returned. \\
\textbf{Description:}
\end{document}
